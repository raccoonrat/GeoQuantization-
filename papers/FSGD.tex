%%%%%%%%%%%%%%%%%%%%%%%%%%%%%%%%%%%%%%%%%%%%%%%%%%%%%%%%%%%%%%%%%%%%%%%%%%%%%%%%
% USENIX Paper Template for "Outlier Geometry and Privacy–Accuracy Duality in LLM Quantization"
%%%%%%%%%%%%%%%%%%%%%%%%%%%%%%%%%%%%%%%%%%%%%%%%%%%%%%%%%%%%%%%%%%%%%%%%%%%%%%%%
\documentclass[letterpaper,twocolumn,10pt]{article}
\usepackage{usenix2019_v3}
%%%%%%%%%%%%%%%%%%%%%%%%%%%%%%%%%%%%%%%%%%%%%%%%%%%%%%%%%%%%%%%%%%%%%%%%%%%%%%%%
% TikZ Illustrations for FSGD & Lattice Framework
%%%%%%%%%%%%%%%%%%%%%%%%%%%%%%%%%%%%%%%%%%%%%%%%%%%%%%%%%%%%%%%%%%%%%%%%%%%%%%%%
\usepackage{tikz}
\usepackage{tikz-3dplot}
\usepackage{xeCJK}
\usetikzlibrary{arrows.meta, positioning, calc, decorations.pathreplacing, shapes.geometric}
% 中文字体设置
\setCJKmainfont{SimSun}

%%%%%%%%%%%%%%%%%%%%%%%%%%%%%%%%%%%%%%%%%%%%%%%%%%%%%%%%%%%%%%%%%%%%%%%%%%%%%%%%
% Figure 1: Functional–Sensitive Geometric Decomposition (FSGD)
%%%%%%%%%%%%%%%%%%%%%%%%%%%%%%%%%%%%%%%%%%%%%%%%%%%%%%%%%%%%%%%%%%%%%%%%%%%%%%%%
\begin{figure*}[t]
\centering
\begin{tikzpicture}[scale=1.2, every node/.style={font=\footnotesize}]
% Main manifold
\shade[ball color=blue!20, opacity=0.5] (0,0) ellipse (3 and 1.5);
\node at (0,-1.8) {\textbf{主功能流形 (Functional Manifold)}};

% High curvature spikes
\foreach \x in {-1.2, 0.4, 1.8} {
  \draw[red!80!black, ultra thick, decorate, decoration={snake, amplitude=2pt, segment length=6pt}] 
  (\x, {1.5 - 0.5*rand}) -- (\x, {2.2 + 0.4*rand});
}
\node[red!70!black] at (1.5,2.5) {尖峰 (High Curvature Spikes)};

% Off-manifold islands
\foreach \pos in {(4,0.8), (-3.5,-0.5), (3,-1.2)}{
  \shade[ball color=orange!20, opacity=0.7] \pos circle (0.6);
}
\node[orange!70!black] at (4.3,1.5) {离岛 (Off-Manifold Islands)};

% Arrows to show flow
\draw[->, thick, gray!70] (-4,0) -- (4.8,0) node[below right]{几何方向 (Geometric Axis)};
\draw[->, thick, gray!70] (0,-2) -- (0,3) node[above]{曲率/敏感性};

% Legend
\node[draw, fill=white, align=left, font=\scriptsize, below right=0.5cm of current bounding box.south west, text width=5cm] (legend) {
\textbf{FSGD 概念:}\\
-- 蓝色椭圆:功能流形\\
-- 橙色圆:敏感离群岛\\
-- 红色曲线:高曲率尖峰
};
\end{tikzpicture}
\caption{Functional–Sensitive Geometric Decomposition (FSGD): 
The LLM weight space exhibits a layered manifold geometry with 
(1) a smooth functional manifold (blue), 
(2) isolated off-manifold islands (orange), and 
(3) high-curvature spikes (red) coupling functionality and sensitivity.}
\label{fig:FSGD}
\end{figure*}

%%%%%%%%%%%%%%%%%%%%%%%%%%%%%%%%%%%%%%%%%%%%%%%%%%%%%%%%%%%%%%%%%%%%%%%%%%%%%%%%
% Figure 2: Perturbation Response Mapping (PRM)
%%%%%%%%%%%%%%%%%%%%%%%%%%%%%%%%%%%%%%%%%%%%%%%%%%%%%%%%%%%%%%%%%%%%%%%%%%%%%%%%
\begin{figure}[t]
\centering
\begin{tikzpicture}[scale=1.1, every node/.style={font=\footnotesize}]
% Axes
\draw[->, thick] (0,0) -- (5,0) node[below right]{\(\Delta \text{PPL}\)};
\draw[->, thick] (0,0) -- (0,4) node[left]{\(\Delta \text{AUC}\)};

% Functional trajectory
\draw[blue!70!black, ultra thick, smooth, tension=0.9]
(0.2,0.3) .. controls (1,0.5) and (3,1.2) .. (4,3);
\node[blue!70!black] at (4.2,2.8) {功能区轨迹};

% Sensitive trajectory
\draw[orange!80!black, ultra thick, dashed, smooth]
(0.3,3.5) .. controls (1.2,3) and (2,2) .. (4,1);
\node[orange!80!black] at (4.1,1.3) {敏感区轨迹};

% Mixed trajectory
\draw[red!70!black, ultra thick, dotted, smooth]
(0.2,0.3) .. controls (1.5,1.8) and (2.5,2.2) .. (4.2,2.5);
\node[red!70!black] at (4.3,2.4) {混合区轨迹};

% Phase labels
\node[font=\scriptsize] at (2,3.8) {隐私改善方向};
\node[font=\scriptsize] at (4.7,0.2) {精度下降方向};
\end{tikzpicture}
\caption{Perturbation Response Mapping (PRM): 
Distinct subspaces (\(\mathcal{W}_{func}, \mathcal{W}_{sens}, \mathcal{W}_{both}\)) 
show separable trajectories in the (\(\Delta \text{PPL}, \Delta \text{AUC}\)) phase space.}
\label{fig:PRM}
\end{figure}

%%%%%%%%%%%%%%%%%%%%%%%%%%%%%%%%%%%%%%%%%%%%%%%%%%%%%%%%%%%%%%%%%%%%%%%%%%%%%%%%
% Figure 3: Lattice–Quantization Geometry
%%%%%%%%%%%%%%%%%%%%%%%%%%%%%%%%%%%%%%%%%%%%%%%%%%%%%%%%%%%%%%%%%%%%%%%%%%%%%%%%
\begin{figure}[t]
\centering
\begin{tikzpicture}[scale=0.9, every node/.style={font=\footnotesize}]
% Lattice grid
\foreach \x in {0,0.6,...,3.6}{
  \foreach \y in {0,0.6,...,3.6}{
    \fill[gray!40] (\x,\y) circle (1pt);
  }
}
% Basis vectors
\draw[->, thick, blue!70!black] (0,0) -- (1.2,0.3) node[below right]{\(\mathbf{b}_1\)};
\draw[->, thick, blue!70!black] (0,0) -- (0.5,1.1) node[left]{\(\mathbf{b}_2\)};
% Target & projection
\fill[red!70!black] (2.8,2.1) circle (2pt) node[above right]{目标向量 \(\mathbf{t}\)};
\fill[orange!80!black] (2.4,2.4) circle (2pt) node[below left]{近似格点 \(\mathbf{v}\)};
\draw[dashed, orange!70!black, thick] (2.8,2.1) -- (2.4,2.4);
% Orthogonalization arrows
\draw[dotted, gray!70] (1.2,0.3) -- (1.2,1.5);
\draw[dotted, gray!70] (0.5,1.1) -- (2.5,1.1);
% Caption text
\node[align=left, font=\scriptsize, below=0.4cm of current bounding box.south]{
蓝色向量形成格基 \(B=[\mathbf{b}_1,\mathbf{b}_2]\)。红点为目标向量 \(\mathbf{t}\),
橙点为 Babai 算法找到的近似格向量 \(\mathbf{v}\)。
误差向量反映格正交性对量化精度的影响。
};
\end{tikzpicture}
\caption{Lattice–Quantization Geometry: Babai projection from target vector 
\(\mathbf{t}\) onto lattice basis vectors \(\mathbf{b}_1, \mathbf{b}_2\). 
Poor orthogonality yields larger quantization error.}
\label{fig:lattice}
\end{figure}

\usepackage{graphicx}
\usepackage{amsmath, amssymb, amsfonts}
\usepackage{booktabs}
\usepackage{multirow}
\usepackage{algorithm}
\usepackage{algorithmic}
\usepackage{url}
\usepackage{xcolor}
\usepackage{hyperref}
\usepackage{caption}
\usepackage{subcaption}
\usepackage{enumitem}

%%%%%%%%%%%%%%%%%%%%%%%%%%%%%%%%%%%%%%%%%%%%%%%%%%%%%%%%%%%%%%%%%%%%%%%%%%%%%%%%
% Title
%%%%%%%%%%%%%%%%%%%%%%%%%%%%%%%%%%%%%%%%%%%%%%%%%%%%%%%%%%%%%%%%%%%%%%%%%%%%%%%%
\begin{document}

\title{\Large \bf Outlier Geometry and Privacy–Accuracy Duality in LLM Quantization: 
A Lattice–Manifold Theoretic Framework}

\author{
{\rm [Your Name]}\\
[Your University]\\
{\tt your.email@domain.com}
}

\maketitle

\begin{abstract}
This paper proposes a geometric–algebraic framework for understanding outlier phenomena in large language model (LLM) quantization. We formulate the dual nature of outliers—functional and sensitive—within a nonlinearly separable manifold geometry, and link the resulting structure to lattice quantization theory through the Babai nearest-plane algorithm. The theoretical insights connect geometric curvature, Hessian eigenspectra, and privacy–accuracy trade-offs, supported by an experimental methodology based on geometric-guided perturbation and lattice stability analysis.
\end{abstract}

%%%%%%%%%%%%%%%%%%%%%%%%%%%%%%%%%%%%%%%%%%%%%%%%%%%%%%%%%%%%%%%%%%%%%%%%%%%%%%%%
\section{Introduction}
%%%%%%%%%%%%%%%%%%%%%%%%%%%%%%%%%%%%%%%%%%%%%%%%%%%%%%%%%%%%%%%%%%%%%%%%%%%%%%%%
LLM quantization offers efficiency but introduces nontrivial accuracy and privacy effects. 
We identify a duality: certain \emph{outlier weights} encode critical generalization (functional outliers), while others memorize sensitive data (sensitive outliers). 
This work builds a unified mathematical bridge between geometric separability and lattice-based quantization error, offering quantitative tools for privacy-aware compression.

%%%%%%%%%%%%%%%%%%%%%%%%%%%%%%%%%%%%%%%%%%%%%%%%%%%%%%%%%%%%%%%%%%%%%%%%%%%%%%%%
\section{Functional–Sensitive Geometric Decomposition (FSGD)}
%%%%%%%%%%%%%%%%%%%%%%%%%%%%%%%%%%%%%%%%%%%%%%%%%%%%%%%%%%%%%%%%%%%%%%%%%%%%%%%%

\subsection{Conceptual Overview}
We hypothesize that the weight space of an LLM forms a \textbf{hierarchical manifold} 
composed of three structural regimes:

\begin{itemize}[leftmargin=2em]
  \item \textbf{Functional Manifold} $\mathcal{W}_{func}$ – smooth low-curvature regions supporting generalizable functions.
  \item \textbf{Off-Manifold Islands} $\mathcal{W}_{sens}$ – isolated, sparse regions memorizing specific data.
  \item \textbf{High-Curvature Spikes} $\mathcal{W}_{both}$ – regions of dual functional and sensitive coupling.
\end{itemize}

Formally:
\[
\mathcal{W} = \mathcal{W}_{func} \cup \mathcal{W}_{sens} \cup \mathcal{W}_{both}
\]

\subsection{Observable Metrics}

\begin{table}[h]
\centering
\caption{Measurable Features of Functional and Sensitive Outlier Structures}
\begin{tabular}{@{}lcccccc@{}}
\toprule
Structure & Type & Curv. $\kappa$ & Hessian Spec. & Sparsity & Align. & Validation \\ 
\midrule
Main Manifold & Func. & Low & Small–Medium & Low & High & PCA \\
Islands & Sens. & Variable & Minor & High & Low & UMAP \\
Spikes & Both & Very High & Outlier & High & Very High & Hessian \\
\bottomrule
\end{tabular}
\end{table}

The separability of these manifolds is non-linear; we use Hessian eigen-decomposition and 
UMAP/t-SNE projection. Fidelity is verified by the \emph{Neighborhood Preservation Ratio} (NPR):

\begin{equation}
NPR = \frac{1}{N}\sum_i \frac{|N_k^{high}(i) \cap N_k^{low}(i)|}{k}
\end{equation}

An NPR above 0.8 indicates reliable embedding.

%%%%%%%%%%%%%%%%%%%%%%%%%%%%%%%%%%%%%%%%%%%%%%%%%%%%%%%%%%%%%%%%%%%%%%%%%%%%%%%%
\section{Geometric-Guided Perturbation Experiment}
%%%%%%%%%%%%%%%%%%%%%%%%%%%%%%%%%%%%%%%%%%%%%%%%%%%%%%%%%%%%%%%%%%%%%%%%%%%%%%%%
We apply Gaussian perturbations to parameter subspaces and observe accuracy (PPL) and privacy (MIA AUC) responses.

\begin{align}
\mathbf{w}'_i &= \mathbf{w}_i + \mathcal{N}(0,\sigma_i^2), \\
\Phi(\sigma_i) &= (\Delta \text{PPL}(\sigma_i), \Delta \text{AUC}(\sigma_i))
\end{align}

Each $\Phi(\sigma_i)$ defines a trajectory in the \emph{Perturbation Response Mapping (PRM)} plane, 
distinguishing $\mathcal{W}_{func}$, $\mathcal{W}_{sens}$, and $\mathcal{W}_{both}$.

\begin{figure}[h]
\centering
%%%%%%%%%%%%%%%%%%%%%%%%%%%%%%%%%%%%%%%%%%%%%%%%%%%%%%%%%%%%%%%%%%%%%%%%%%%%%%%%
% TikZ Illustrations for FSGD & Lattice Framework
%%%%%%%%%%%%%%%%%%%%%%%%%%%%%%%%%%%%%%%%%%%%%%%%%%%%%%%%%%%%%%%%%%%%%%%%%%%%%%%%
\usepackage{tikz}
\usepackage{tikz-3dplot}
\usepackage{xeCJK}
\usetikzlibrary{arrows.meta, positioning, calc, decorations.pathreplacing, shapes.geometric}
% 中文字体设置
\setCJKmainfont{SimSun}

%%%%%%%%%%%%%%%%%%%%%%%%%%%%%%%%%%%%%%%%%%%%%%%%%%%%%%%%%%%%%%%%%%%%%%%%%%%%%%%%
% Figure 1: Functional–Sensitive Geometric Decomposition (FSGD)
%%%%%%%%%%%%%%%%%%%%%%%%%%%%%%%%%%%%%%%%%%%%%%%%%%%%%%%%%%%%%%%%%%%%%%%%%%%%%%%%
\begin{figure*}[t]
\centering
\begin{tikzpicture}[scale=1.2, every node/.style={font=\footnotesize}]
% Main manifold
\shade[ball color=blue!20, opacity=0.5] (0,0) ellipse (3 and 1.5);
\node at (0,-1.8) {\textbf{主功能流形 (Functional Manifold)}};

% High curvature spikes
\foreach \x in {-1.2, 0.4, 1.8} {
  \draw[red!80!black, ultra thick, decorate, decoration={snake, amplitude=2pt, segment length=6pt}] 
  (\x, {1.5 - 0.5*rand}) -- (\x, {2.2 + 0.4*rand});
}
\node[red!70!black] at (1.5,2.5) {尖峰 (High Curvature Spikes)};

% Off-manifold islands
\foreach \pos in {(4,0.8), (-3.5,-0.5), (3,-1.2)}{
  \shade[ball color=orange!20, opacity=0.7] \pos circle (0.6);
}
\node[orange!70!black] at (4.3,1.5) {离岛 (Off-Manifold Islands)};

% Arrows to show flow
\draw[->, thick, gray!70] (-4,0) -- (4.8,0) node[below right]{几何方向 (Geometric Axis)};
\draw[->, thick, gray!70] (0,-2) -- (0,3) node[above]{曲率/敏感性};

% Legend
\node[draw, fill=white, align=left, font=\scriptsize, below right=0.5cm of current bounding box.south west, text width=5cm] (legend) {
\textbf{FSGD 概念:}\\
-- 蓝色椭圆:功能流形\\
-- 橙色圆:敏感离群岛\\
-- 红色曲线:高曲率尖峰
};
\end{tikzpicture}
\caption{Functional–Sensitive Geometric Decomposition (FSGD): 
The LLM weight space exhibits a layered manifold geometry with 
(1) a smooth functional manifold (blue), 
(2) isolated off-manifold islands (orange), and 
(3) high-curvature spikes (red) coupling functionality and sensitivity.}
\label{fig:FSGD}
\end{figure*}

%%%%%%%%%%%%%%%%%%%%%%%%%%%%%%%%%%%%%%%%%%%%%%%%%%%%%%%%%%%%%%%%%%%%%%%%%%%%%%%%
% Figure 2: Perturbation Response Mapping (PRM)
%%%%%%%%%%%%%%%%%%%%%%%%%%%%%%%%%%%%%%%%%%%%%%%%%%%%%%%%%%%%%%%%%%%%%%%%%%%%%%%%
\begin{figure}[t]
\centering
\begin{tikzpicture}[scale=1.1, every node/.style={font=\footnotesize}]
% Axes
\draw[->, thick] (0,0) -- (5,0) node[below right]{\(\Delta \text{PPL}\)};
\draw[->, thick] (0,0) -- (0,4) node[left]{\(\Delta \text{AUC}\)};

% Functional trajectory
\draw[blue!70!black, ultra thick, smooth, tension=0.9]
(0.2,0.3) .. controls (1,0.5) and (3,1.2) .. (4,3);
\node[blue!70!black] at (4.2,2.8) {功能区轨迹};

% Sensitive trajectory
\draw[orange!80!black, ultra thick, dashed, smooth]
(0.3,3.5) .. controls (1.2,3) and (2,2) .. (4,1);
\node[orange!80!black] at (4.1,1.3) {敏感区轨迹};

% Mixed trajectory
\draw[red!70!black, ultra thick, dotted, smooth]
(0.2,0.3) .. controls (1.5,1.8) and (2.5,2.2) .. (4.2,2.5);
\node[red!70!black] at (4.3,2.4) {混合区轨迹};

% Phase labels
\node[font=\scriptsize] at (2,3.8) {隐私改善方向};
\node[font=\scriptsize] at (4.7,0.2) {精度下降方向};
\end{tikzpicture}
\caption{Perturbation Response Mapping (PRM): 
Distinct subspaces (\(\mathcal{W}_{func}, \mathcal{W}_{sens}, \mathcal{W}_{both}\)) 
show separable trajectories in the (\(\Delta \text{PPL}, \Delta \text{AUC}\)) phase space.}
\label{fig:PRM}
\end{figure}

%%%%%%%%%%%%%%%%%%%%%%%%%%%%%%%%%%%%%%%%%%%%%%%%%%%%%%%%%%%%%%%%%%%%%%%%%%%%%%%%
% Figure 3: Lattice–Quantization Geometry
%%%%%%%%%%%%%%%%%%%%%%%%%%%%%%%%%%%%%%%%%%%%%%%%%%%%%%%%%%%%%%%%%%%%%%%%%%%%%%%%
\begin{figure}[t]
\centering
\begin{tikzpicture}[scale=0.9, every node/.style={font=\footnotesize}]
% Lattice grid
\foreach \x in {0,0.6,...,3.6}{
  \foreach \y in {0,0.6,...,3.6}{
    \fill[gray!40] (\x,\y) circle (1pt);
  }
}
% Basis vectors
\draw[->, thick, blue!70!black] (0,0) -- (1.2,0.3) node[below right]{\(\mathbf{b}_1\)};
\draw[->, thick, blue!70!black] (0,0) -- (0.5,1.1) node[left]{\(\mathbf{b}_2\)};
% Target & projection
\fill[red!70!black] (2.8,2.1) circle (2pt) node[above right]{目标向量 \(\mathbf{t}\)};
\fill[orange!80!black] (2.4,2.4) circle (2pt) node[below left]{近似格点 \(\mathbf{v}\)};
\draw[dashed, orange!70!black, thick] (2.8,2.1) -- (2.4,2.4);
% Orthogonalization arrows
\draw[dotted, gray!70] (1.2,0.3) -- (1.2,1.5);
\draw[dotted, gray!70] (0.5,1.1) -- (2.5,1.1);
% Caption text
\node[align=left, font=\scriptsize, below=0.4cm of current bounding box.south]{
蓝色向量形成格基 \(B=[\mathbf{b}_1,\mathbf{b}_2]\)。红点为目标向量 \(\mathbf{t}\),
橙点为 Babai 算法找到的近似格向量 \(\mathbf{v}\)。
误差向量反映格正交性对量化精度的影响。
};
\end{tikzpicture}
\caption{Lattice–Quantization Geometry: Babai projection from target vector 
\(\mathbf{t}\) onto lattice basis vectors \(\mathbf{b}_1, \mathbf{b}_2\). 
Poor orthogonality yields larger quantization error.}
\label{fig:lattice}
\end{figure}

\caption{Illustration of perturbation response trajectories for functional, sensitive, and mixed outlier subspaces.}
\end{figure}

%%%%%%%%%%%%%%%%%%%%%%%%%%%%%%%%%%%%%%%%%%%%%%%%%%%%%%%%%%%%%%%%%%%%%%%%%%%%%%%%
\section{GPTQ Lattice-Theoretic Framework}
%%%%%%%%%%%%%%%%%%%%%%%%%%%%%%%%%%%%%%%%%%%%%%%%%%%%%%%%%%%%%%%%%%%%%%%%%%%%%%%%
Quantization is reformulated as a lattice nearest vector problem (CVP).  
For a linear layer:
\[
Y = XW, \quad X \in \mathbb{R}^{n\times c}, \quad W \in \mathbb{R}^{c\times d}
\]
We minimize:
\begin{equation}
\min_{\mathbf{z}\in \mathbb{Z}^c} \|X\mathbf{w} - sX\mathbf{z}\|_2^2
\end{equation}
The lattice basis:
\[
B = sX
\]

%%%%%%%%%%%%%%%%%%%%%%%%%%%%%%%%%%%%%%%%%%%%%%%%%%%%%%%%%%%%%%%%%%%%%%%%%%%%%%%%
\subsection{QR Decomposition and Numerical Stability}
We compute a QR factorization:
\[
B = QR, \quad Q\in\mathbb{R}^{n\times c}, R\in\mathbb{R}^{c\times c}
\]
Babai’s algorithm operates in the $R$-space, ensuring numerical stability.

%%%%%%%%%%%%%%%%%%%%%%%%%%%%%%%%%%%%%%%%%%%%%%%%%%%%%%%%%%%%%%%%%%%%%%%%%%%%%%%%
\subsection{Babai Error Bound}
\begin{equation}
\|\mathbf{v}-\mathbf{t}\|_2^2 \le \frac{1}{4}\sum_{i=1}^{c}\|\mathbf{b}^*_i\|_2^2
\end{equation}
where $\mathbf{b}_i^*$ are Gram–Schmidt orthogonalized basis vectors.

We define:
\[
\eta = \frac{1}{c}\sum_i\|\mathbf{b}^*_i\|_2, \quad 
\mathcal{S} = \frac{1}{1+\text{Var}_X[\kappa(B)]}
\]
representing the \textit{Lattice Orthogonality Tensor Norm (LOTN)} and \textit{Lattice Stability Index (LSI)} respectively.

%%%%%%%%%%%%%%%%%%%%%%%%%%%%%%%%%%%%%%%%%%%%%%%%%%%%%%%%%%%%%%%%%%%%%%%%%%%%%%%%
\subsection{Sample Dependency and Stability}
Bootstrapped calibration subsets $X_i$ yield $B_i = sX_i$ and corresponding condition numbers $\kappa(B_i)$.
Variance of $\kappa(B_i)$ quantifies geometry stability.
A higher $\mathcal{S}$ (closer to 1) indicates robust lattice structure.

%%%%%%%%%%%%%%%%%%%%%%%%%%%%%%%%%%%%%%%%%%%%%%%%%%%%%%%%%%%%%%%%%%%%%%%%%%%%%%%%
\subsection{Empirical Verification}
We expect a strong correlation between $\log(\kappa(B))$ and $\log(MSE)$ across layers:
\[
\rho = \text{Spearman}(\log(\kappa(B)), \log(MSE)) > 0.7
\]
\begin{figure}[h]
\centering
%%%%%%%%%%%%%%%%%%%%%%%%%%%%%%%%%%%%%%%%%%%%%%%%%%%%%%%%%%%%%%%%%%%%%%%%%%%%%%%%
% TikZ Illustrations for FSGD & Lattice Framework
%%%%%%%%%%%%%%%%%%%%%%%%%%%%%%%%%%%%%%%%%%%%%%%%%%%%%%%%%%%%%%%%%%%%%%%%%%%%%%%%
\usepackage{tikz}
\usepackage{tikz-3dplot}
\usepackage{xeCJK}
\usetikzlibrary{arrows.meta, positioning, calc, decorations.pathreplacing, shapes.geometric}
% 中文字体设置
\setCJKmainfont{SimSun}

%%%%%%%%%%%%%%%%%%%%%%%%%%%%%%%%%%%%%%%%%%%%%%%%%%%%%%%%%%%%%%%%%%%%%%%%%%%%%%%%
% Figure 1: Functional–Sensitive Geometric Decomposition (FSGD)
%%%%%%%%%%%%%%%%%%%%%%%%%%%%%%%%%%%%%%%%%%%%%%%%%%%%%%%%%%%%%%%%%%%%%%%%%%%%%%%%
\begin{figure*}[t]
\centering
\begin{tikzpicture}[scale=1.2, every node/.style={font=\footnotesize}]
% Main manifold
\shade[ball color=blue!20, opacity=0.5] (0,0) ellipse (3 and 1.5);
\node at (0,-1.8) {\textbf{主功能流形 (Functional Manifold)}};

% High curvature spikes
\foreach \x in {-1.2, 0.4, 1.8} {
  \draw[red!80!black, ultra thick, decorate, decoration={snake, amplitude=2pt, segment length=6pt}] 
  (\x, {1.5 - 0.5*rand}) -- (\x, {2.2 + 0.4*rand});
}
\node[red!70!black] at (1.5,2.5) {尖峰 (High Curvature Spikes)};

% Off-manifold islands
\foreach \pos in {(4,0.8), (-3.5,-0.5), (3,-1.2)}{
  \shade[ball color=orange!20, opacity=0.7] \pos circle (0.6);
}
\node[orange!70!black] at (4.3,1.5) {离岛 (Off-Manifold Islands)};

% Arrows to show flow
\draw[->, thick, gray!70] (-4,0) -- (4.8,0) node[below right]{几何方向 (Geometric Axis)};
\draw[->, thick, gray!70] (0,-2) -- (0,3) node[above]{曲率/敏感性};

% Legend
\node[draw, fill=white, align=left, font=\scriptsize, below right=0.5cm of current bounding box.south west, text width=5cm] (legend) {
\textbf{FSGD 概念:}\\
-- 蓝色椭圆:功能流形\\
-- 橙色圆:敏感离群岛\\
-- 红色曲线:高曲率尖峰
};
\end{tikzpicture}
\caption{Functional–Sensitive Geometric Decomposition (FSGD): 
The LLM weight space exhibits a layered manifold geometry with 
(1) a smooth functional manifold (blue), 
(2) isolated off-manifold islands (orange), and 
(3) high-curvature spikes (red) coupling functionality and sensitivity.}
\label{fig:FSGD}
\end{figure*}

%%%%%%%%%%%%%%%%%%%%%%%%%%%%%%%%%%%%%%%%%%%%%%%%%%%%%%%%%%%%%%%%%%%%%%%%%%%%%%%%
% Figure 2: Perturbation Response Mapping (PRM)
%%%%%%%%%%%%%%%%%%%%%%%%%%%%%%%%%%%%%%%%%%%%%%%%%%%%%%%%%%%%%%%%%%%%%%%%%%%%%%%%
\begin{figure}[t]
\centering
\begin{tikzpicture}[scale=1.1, every node/.style={font=\footnotesize}]
% Axes
\draw[->, thick] (0,0) -- (5,0) node[below right]{\(\Delta \text{PPL}\)};
\draw[->, thick] (0,0) -- (0,4) node[left]{\(\Delta \text{AUC}\)};

% Functional trajectory
\draw[blue!70!black, ultra thick, smooth, tension=0.9]
(0.2,0.3) .. controls (1,0.5) and (3,1.2) .. (4,3);
\node[blue!70!black] at (4.2,2.8) {功能区轨迹};

% Sensitive trajectory
\draw[orange!80!black, ultra thick, dashed, smooth]
(0.3,3.5) .. controls (1.2,3) and (2,2) .. (4,1);
\node[orange!80!black] at (4.1,1.3) {敏感区轨迹};

% Mixed trajectory
\draw[red!70!black, ultra thick, dotted, smooth]
(0.2,0.3) .. controls (1.5,1.8) and (2.5,2.2) .. (4.2,2.5);
\node[red!70!black] at (4.3,2.4) {混合区轨迹};

% Phase labels
\node[font=\scriptsize] at (2,3.8) {隐私改善方向};
\node[font=\scriptsize] at (4.7,0.2) {精度下降方向};
\end{tikzpicture}
\caption{Perturbation Response Mapping (PRM): 
Distinct subspaces (\(\mathcal{W}_{func}, \mathcal{W}_{sens}, \mathcal{W}_{both}\)) 
show separable trajectories in the (\(\Delta \text{PPL}, \Delta \text{AUC}\)) phase space.}
\label{fig:PRM}
\end{figure}

%%%%%%%%%%%%%%%%%%%%%%%%%%%%%%%%%%%%%%%%%%%%%%%%%%%%%%%%%%%%%%%%%%%%%%%%%%%%%%%%
% Figure 3: Lattice–Quantization Geometry
%%%%%%%%%%%%%%%%%%%%%%%%%%%%%%%%%%%%%%%%%%%%%%%%%%%%%%%%%%%%%%%%%%%%%%%%%%%%%%%%
\begin{figure}[t]
\centering
\begin{tikzpicture}[scale=0.9, every node/.style={font=\footnotesize}]
% Lattice grid
\foreach \x in {0,0.6,...,3.6}{
  \foreach \y in {0,0.6,...,3.6}{
    \fill[gray!40] (\x,\y) circle (1pt);
  }
}
% Basis vectors
\draw[->, thick, blue!70!black] (0,0) -- (1.2,0.3) node[below right]{\(\mathbf{b}_1\)};
\draw[->, thick, blue!70!black] (0,0) -- (0.5,1.1) node[left]{\(\mathbf{b}_2\)};
% Target & projection
\fill[red!70!black] (2.8,2.1) circle (2pt) node[above right]{目标向量 \(\mathbf{t}\)};
\fill[orange!80!black] (2.4,2.4) circle (2pt) node[below left]{近似格点 \(\mathbf{v}\)};
\draw[dashed, orange!70!black, thick] (2.8,2.1) -- (2.4,2.4);
% Orthogonalization arrows
\draw[dotted, gray!70] (1.2,0.3) -- (1.2,1.5);
\draw[dotted, gray!70] (0.5,1.1) -- (2.5,1.1);
% Caption text
\node[align=left, font=\scriptsize, below=0.4cm of current bounding box.south]{
蓝色向量形成格基 \(B=[\mathbf{b}_1,\mathbf{b}_2]\)。红点为目标向量 \(\mathbf{t}\),
橙点为 Babai 算法找到的近似格向量 \(\mathbf{v}\)。
误差向量反映格正交性对量化精度的影响。
};
\end{tikzpicture}
\caption{Lattice–Quantization Geometry: Babai projection from target vector 
\(\mathbf{t}\) onto lattice basis vectors \(\mathbf{b}_1, \mathbf{b}_2\). 
Poor orthogonality yields larger quantization error.}
\label{fig:lattice}
\end{figure}

\caption{Hypothesized correlation between lattice condition number and quantization error across layers.}
\end{figure}

%%%%%%%%%%%%%%%%%%%%%%%%%%%%%%%%%%%%%%%%%%%%%%%%%%%%%%%%%%%%%%%%%%%%%%%%%%%%%%%%
\section{Experimental Implementation Details}
%%%%%%%%%%%%%%%%%%%%%%%%%%%%%%%%%%%%%%%%%%%%%%%%%%%%%%%%%%%%%%%%%%%%%%%%%%%%%%%%
\begin{table}[h]
\centering
\caption{Implementation Summary of Geometric–Lattice Framework}
\begin{tabular}{@{}lcccc@{}}
\toprule
Module & Tool & Input & Output & Metric \\ 
\midrule
Hessian Spectrum & PyTorch + Lanczos & Weights & Eigenvalues & Curv. \\
UMAP Cluster & UMAP-learn & Activations & 2D Embedding & NPR \\
Noise Injection & NumPy + PyTorch & Params & ΔPPL/ΔAUC & PRM \\
Lattice Compute & SciPy.linalg.qr & X & B= sX, R & κ(B), LSI \\
Visualization & Matplotlib & Logs & Phase Plots & Correlation \\
\bottomrule
\end{tabular}
\end{table}

%%%%%%%%%%%%%%%%%%%%%%%%%%%%%%%%%%%%%%%%%%%%%%%%%%%%%%%%%%%%%%%%%%%%%%%%%%%%%%%%
\section{Conclusion}
%%%%%%%%%%%%%%%%%%%%%%%%%%%%%%%%%%%%%%%%%%%%%%%%%%%%%%%%%%%%%%%%%%%%%%%%%%%%%%%%
We developed a unified framework that bridges the geometry of outliers and the algebra of quantization.
The Functional–Sensitive Geometric Decomposition (FSGD) and Lattice-Theoretic Extension of GPTQ provide 
a path to understand and mitigate the privacy–accuracy duality in model compression.

%%%%%%%%%%%%%%%%%%%%%%%%%%%%%%%%%%%%%%%%%%%%%%%%%%%%%%%%%%%%%%%%%%%%%%%%%%%%%%%%
\bibliographystyle{plain}
\bibliography{references}

\end{document}
