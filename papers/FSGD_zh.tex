%%%%%%%%%%%%%%%%%%%%%%%%%%%%%%%%%%%%%%%%%%%%%%%%%%%%%%%%%%%%%%%%%%%%%%%%%%%%%%%%
% USENIX Style Paper + TikZ + Python 数据绘图示例
%%%%%%%%%%%%%%%%%%%%%%%%%%%%%%%%%%%%%%%%%%%%%%%%%%%%%%%%%%%%%%%%%%%%%%%%%%%%%%%%
%!TEX encoding = UTF-8 Unicode
\documentclass[letterpaper,twocolumn,10pt]{article}
\usepackage[utf8]{inputenc}
% \usepackage{usenix2019_v3}  % 暂时注释掉以测试中文支持

%%%%%%%%%%%%%%%%%%%%%%%%%%%%%%%%%%%%%%%%%%%%%%%%%%%%%%%%%%%%%%%%%%%%%%%%%%%%%%%%
% 常用宏包
%%%%%%%%%%%%%%%%%%%%%%%%%%%%%%%%%%%%%%%%%%%%%%%%%%%%%%%%%%%%%%%%%%%%%%%%%%%%%%%%
\usepackage{graphicx}
\usepackage{amsmath, amssymb, amsfonts}
\usepackage{booktabs}
\usepackage{multirow}
\usepackage{tikz}
\usepackage{pgfplots}
\usepackage{pythontex}
\usepackage{xcolor}
\usepackage{xeCJK}
\usepackage{caption}
\usepackage{subcaption}
% 中文字体设置 - 简化配置
\setCJKmainfont{SimSun}

%%%%%%%%%%%%%%%%%%%%%%%%%%%%%%%%%%%%%%%%%%%%%%%%%%%%%%%%%%%%%%%%%%%%%%%%%%%%%%%%
% PGFPlots 设置
%%%%%%%%%%%%%%%%%%%%%%%%%%%%%%%%%%%%%%%%%%%%%%%%%%%%%%%%%%%%%%%%%%%%%%%%%%%%%%%%
\pgfplotsset{compat=1.18}
\usetikzlibrary{arrows.meta, positioning, decorations.pathreplacing}

%%%%%%%%%%%%%%%%%%%%%%%%%%%%%%%%%%%%%%%%%%%%%%%%%%%%%%%%%%%%%%%%%%%%%%%%%%%%%%%%
\begin{document}

\title{几何引导扰动相图:LLM量化中精度隐私双变量实验示意}
\author{
{\rm 示例作者}\\
[机构名称] \\
{\tt example@email.com}
}
\maketitle

%%%%%%%%%%%%%%%%%%%%%%%%%%%%%%%%%%%%%%%%%%%%%%%%%%%%%%%%%%%%%%%%%%%%%%%%%%%%%%%%
\begin{abstract}
本文展示了“几何引导扰动相图 (Perturbation Response Mapping, PRM)”的可视化实现,
通过 Python 自动生成模拟实验数据,并使用 TikZ/PGFPlots 绘制三类参数子空间
(功能区、敏感区、混合区)在 $(\Delta \mathrm{PPL}, \Delta \mathrm{AUC})$ 平面上的轨迹。
该图示体现了 LLM 量化中精度损失与隐私提升的非线性耦合关系。
\end{abstract}

%%%%%%%%%%%%%%%%%%%%%%%%%%%%%%%%%%%%%%%%%%%%%%%%%%%%%%%%%%%%%%%%%%%%%%%%%%%%%%%%
\section{引言}
在大语言模型的量化过程中,权重扰动会导致精度下降(困惑度 $\mathrm{PPL}$ 上升)、
同时可能降低隐私攻击的可行性(成员推理攻击 AUC 下降)。
本文基于理论框架“功能–敏感几何分解 (FSGD)”构建一个二维相图,
展示不同子空间的扰动响应特征。

%%%%%%%%%%%%%%%%%%%%%%%%%%%%%%%%%%%%%%%%%%%%%%%%%%%%%%%%%%%%%%%%%%%%%%%%%%%%%%%%
\section{Python 数据生成}
我们使用 \texttt{pythontex} 在编译过程中生成模拟数据。
每条曲线代表一种参数子空间的扰动响应:

\begin{pycode}
import numpy as np
import pandas as pd

np.random.seed(42)
sigma = np.linspace(0, 1, 15)

# 模拟三类曲线:功能区、敏感区、混合区
ppl_func = 5 * sigma**1.8 + np.random.normal(0, 0.1, len(sigma))
auc_func = 1 - 0.1*sigma + np.random.normal(0, 0.02, len(sigma))

ppl_sens = 2 * sigma + np.random.normal(0, 0.05, len(sigma))
auc_sens = 1 - 0.5*sigma**1.5 + np.random.normal(0, 0.02, len(sigma))

ppl_both = 3 * sigma**1.2 + np.random.normal(0, 0.08, len(sigma))
auc_both = 1 - 0.3*sigma + np.random.normal(0, 0.03, len(sigma))

df = pd.DataFrame({
    "sigma": sigma,
    "ppl_func": ppl_func,
    "auc_func": auc_func,
    "ppl_sens": ppl_sens,
    "auc_sens": auc_sens,
    "ppl_both": ppl_both,
    "auc_both": auc_both
})
df.to_csv("prm_data.csv", index=False)
\end{pycode}

该脚本在第一次编译时生成 `prm_data.csv`,用于后续 TikZ 绘图。

%%%%%%%%%%%%%%%%%%%%%%%%%%%%%%%%%%%%%%%%%%%%%%%%%%%%%%%%%%%%%%%%%%%%%%%%%%%%%%%%
\section{PRM 实验相图}
图 \ref{fig:prm} 展示了实验模拟结果。
横轴为困惑度变化 $\Delta \mathrm{PPL}$(越大代表精度下降),
纵轴为成员推理攻击 AUC 的变化 $\Delta \mathrm{AUC}$(越大代表隐私风险上升)。

\begin{figure}[h]
\centering
\begin{tikzpicture}
\begin{axis}[
    width=0.95\linewidth,
    height=6cm,
    xlabel={$\Delta \mathrm{PPL}$ (精度变化)},
    ylabel={$\Delta \mathrm{AUC}$ (隐私变化)},
    xmin=0, xmax=5.5,
    ymin=0.4, ymax=1.05,
    grid=major,
    legend style={at={(0.02,0.98)},anchor=north west,font=\scriptsize},
    label style={font=\small},
    ticklabel style={font=\scriptsize},
    every axis plot/.append style={thick, mark=*},
]

% 加载 CSV 数据
\addplot[blue!70!black, mark=*, mark options={fill=blue}, line width=1pt]
    table[x=ppl_func, y=auc_func, col sep=comma] {prm_data.csv};
\addlegendentry{功能区(Functional)}

\addplot[orange!80!black, mark=triangle*, mark options={fill=orange}, dashed, line width=1pt]
    table[x=ppl_sens, y=auc_sens, col sep=comma] {prm_data.csv};
\addlegendentry{敏感区(Sensitive)}

\addplot[red!70!black, mark=square*, mark options={fill=red}, dotted, line width=1pt]
    table[x=ppl_both, y=auc_both, col sep=comma] {prm_data.csv};
\addlegendentry{混合区(Mixed)}

\end{axis}
\end{tikzpicture}
\caption{几何引导扰动相图(PRM):三类子空间在 $(\Delta \mathrm{PPL}, \Delta \mathrm{AUC})$ 平面上的响应轨迹。
功能区主要表现为显著精度下降;敏感区呈现隐私改善趋势;
混合区位于两者之间。}
\label{fig:prm}
\end{figure}

%%%%%%%%%%%%%%%%%%%%%%%%%%%%%%%%%%%%%%%%%%%%%%%%%%%%%%%%%%%%%%%%%%%%%%%%%%%%%%%%
\section{结果讨论}
从图中可见:
\begin{itemize}
  \item 功能区(蓝线)随扰动增强,精度下降显著,但隐私变化较小;
  \item 敏感区(橙线)表现为隐私快速改善(AUC 降低),精度影响有限;
  \item 混合区(红线)位于中间,反映两种效应叠加。
\end{itemize}

该结果验证了“功能–敏感几何分解 (FSGD)”理论中的可分性假设,
并展示了量化噪声对隐私与精度的非线性影响路径。

%%%%%%%%%%%%%%%%%%%%%%%%%%%%%%%%%%%%%%%%%%%%%%%%%%%%%%%%%%%%%%%%%%%%%%%%%%%%%%%%
% 包含图表文件
%%%%%%%%%%%%%%%%%%%%%%%%%%%%%%%%%%%%%%%%%%%%%%%%%%%%%%%%%%%%%%%%%%%%%%%%%%%%%%%%
%%%%%%%%%%%%%%%%%%%%%%%%%%%%%%%%%%%%%%%%%%%%%%%%%%%%%%%%%%%%%%%%%%%%%%%%%%%%%%%%
% TikZ Illustrations for FSGD & Lattice Framework
%%%%%%%%%%%%%%%%%%%%%%%%%%%%%%%%%%%%%%%%%%%%%%%%%%%%%%%%%%%%%%%%%%%%%%%%%%%%%%%%
\usepackage{tikz}
\usepackage{tikz-3dplot}
\usepackage{xeCJK}
\usetikzlibrary{arrows.meta, positioning, calc, decorations.pathreplacing, shapes.geometric}
% 中文字体设置
\setCJKmainfont{SimSun}

%%%%%%%%%%%%%%%%%%%%%%%%%%%%%%%%%%%%%%%%%%%%%%%%%%%%%%%%%%%%%%%%%%%%%%%%%%%%%%%%
% Figure 1: Functional–Sensitive Geometric Decomposition (FSGD)
%%%%%%%%%%%%%%%%%%%%%%%%%%%%%%%%%%%%%%%%%%%%%%%%%%%%%%%%%%%%%%%%%%%%%%%%%%%%%%%%
\begin{figure*}[t]
\centering
\begin{tikzpicture}[scale=1.2, every node/.style={font=\footnotesize}]
% Main manifold
\shade[ball color=blue!20, opacity=0.5] (0,0) ellipse (3 and 1.5);
\node at (0,-1.8) {\textbf{主功能流形 (Functional Manifold)}};

% High curvature spikes
\foreach \x in {-1.2, 0.4, 1.8} {
  \draw[red!80!black, ultra thick, decorate, decoration={snake, amplitude=2pt, segment length=6pt}] 
  (\x, {1.5 - 0.5*rand}) -- (\x, {2.2 + 0.4*rand});
}
\node[red!70!black] at (1.5,2.5) {尖峰 (High Curvature Spikes)};

% Off-manifold islands
\foreach \pos in {(4,0.8), (-3.5,-0.5), (3,-1.2)}{
  \shade[ball color=orange!20, opacity=0.7] \pos circle (0.6);
}
\node[orange!70!black] at (4.3,1.5) {离岛 (Off-Manifold Islands)};

% Arrows to show flow
\draw[->, thick, gray!70] (-4,0) -- (4.8,0) node[below right]{几何方向 (Geometric Axis)};
\draw[->, thick, gray!70] (0,-2) -- (0,3) node[above]{曲率/敏感性};

% Legend
\node[draw, fill=white, align=left, font=\scriptsize, below right=0.5cm of current bounding box.south west, text width=5cm] (legend) {
\textbf{FSGD 概念:}\\
-- 蓝色椭圆:功能流形\\
-- 橙色圆:敏感离群岛\\
-- 红色曲线:高曲率尖峰
};
\end{tikzpicture}
\caption{Functional–Sensitive Geometric Decomposition (FSGD): 
The LLM weight space exhibits a layered manifold geometry with 
(1) a smooth functional manifold (blue), 
(2) isolated off-manifold islands (orange), and 
(3) high-curvature spikes (red) coupling functionality and sensitivity.}
\label{fig:FSGD}
\end{figure*}

%%%%%%%%%%%%%%%%%%%%%%%%%%%%%%%%%%%%%%%%%%%%%%%%%%%%%%%%%%%%%%%%%%%%%%%%%%%%%%%%
% Figure 2: Perturbation Response Mapping (PRM)
%%%%%%%%%%%%%%%%%%%%%%%%%%%%%%%%%%%%%%%%%%%%%%%%%%%%%%%%%%%%%%%%%%%%%%%%%%%%%%%%
\begin{figure}[t]
\centering
\begin{tikzpicture}[scale=1.1, every node/.style={font=\footnotesize}]
% Axes
\draw[->, thick] (0,0) -- (5,0) node[below right]{\(\Delta \text{PPL}\)};
\draw[->, thick] (0,0) -- (0,4) node[left]{\(\Delta \text{AUC}\)};

% Functional trajectory
\draw[blue!70!black, ultra thick, smooth, tension=0.9]
(0.2,0.3) .. controls (1,0.5) and (3,1.2) .. (4,3);
\node[blue!70!black] at (4.2,2.8) {功能区轨迹};

% Sensitive trajectory
\draw[orange!80!black, ultra thick, dashed, smooth]
(0.3,3.5) .. controls (1.2,3) and (2,2) .. (4,1);
\node[orange!80!black] at (4.1,1.3) {敏感区轨迹};

% Mixed trajectory
\draw[red!70!black, ultra thick, dotted, smooth]
(0.2,0.3) .. controls (1.5,1.8) and (2.5,2.2) .. (4.2,2.5);
\node[red!70!black] at (4.3,2.4) {混合区轨迹};

% Phase labels
\node[font=\scriptsize] at (2,3.8) {隐私改善方向};
\node[font=\scriptsize] at (4.7,0.2) {精度下降方向};
\end{tikzpicture}
\caption{Perturbation Response Mapping (PRM): 
Distinct subspaces (\(\mathcal{W}_{func}, \mathcal{W}_{sens}, \mathcal{W}_{both}\)) 
show separable trajectories in the (\(\Delta \text{PPL}, \Delta \text{AUC}\)) phase space.}
\label{fig:PRM}
\end{figure}

%%%%%%%%%%%%%%%%%%%%%%%%%%%%%%%%%%%%%%%%%%%%%%%%%%%%%%%%%%%%%%%%%%%%%%%%%%%%%%%%
% Figure 3: Lattice–Quantization Geometry
%%%%%%%%%%%%%%%%%%%%%%%%%%%%%%%%%%%%%%%%%%%%%%%%%%%%%%%%%%%%%%%%%%%%%%%%%%%%%%%%
\begin{figure}[t]
\centering
\begin{tikzpicture}[scale=0.9, every node/.style={font=\footnotesize}]
% Lattice grid
\foreach \x in {0,0.6,...,3.6}{
  \foreach \y in {0,0.6,...,3.6}{
    \fill[gray!40] (\x,\y) circle (1pt);
  }
}
% Basis vectors
\draw[->, thick, blue!70!black] (0,0) -- (1.2,0.3) node[below right]{\(\mathbf{b}_1\)};
\draw[->, thick, blue!70!black] (0,0) -- (0.5,1.1) node[left]{\(\mathbf{b}_2\)};
% Target & projection
\fill[red!70!black] (2.8,2.1) circle (2pt) node[above right]{目标向量 \(\mathbf{t}\)};
\fill[orange!80!black] (2.4,2.4) circle (2pt) node[below left]{近似格点 \(\mathbf{v}\)};
\draw[dashed, orange!70!black, thick] (2.8,2.1) -- (2.4,2.4);
% Orthogonalization arrows
\draw[dotted, gray!70] (1.2,0.3) -- (1.2,1.5);
\draw[dotted, gray!70] (0.5,1.1) -- (2.5,1.1);
% Caption text
\node[align=left, font=\scriptsize, below=0.4cm of current bounding box.south]{
蓝色向量形成格基 \(B=[\mathbf{b}_1,\mathbf{b}_2]\)。红点为目标向量 \(\mathbf{t}\),
橙点为 Babai 算法找到的近似格向量 \(\mathbf{v}\)。
误差向量反映格正交性对量化精度的影响。
};
\end{tikzpicture}
\caption{Lattice–Quantization Geometry: Babai projection from target vector 
\(\mathbf{t}\) onto lattice basis vectors \(\mathbf{b}_1, \mathbf{b}_2\). 
Poor orthogonality yields larger quantization error.}
\label{fig:lattice}
\end{figure}


%%%%%%%%%%%%%%%%%%%%%%%%%%%%%%%%%%%%%%%%%%%%%%%%%%%%%%%%%%%%%%%%%%%%%%%%%%%%%%%%
\section{结论}
本示例展示了如何结合 Python 数据生成与 LaTeX 的 TikZ/PGFPlots 绘图,
在论文编译过程中直接产生科学可视化图表。
实际实验可替换模拟数据,以呈现真实 LLM 量化实验结果。

%%%%%%%%%%%%%%%%%%%%%%%%%%%%%%%%%%%%%%%%%%%%%%%%%%%%%%%%%%%%%%%%%%%%%%%%%%%%%%%%
\end{document}
